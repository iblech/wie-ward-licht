\documentclass[twoside]{zirkelblatt1415}
\usepackage{mathtools}
\usepackage{wrapfig}
\usepackage{tabto}
\usepackage{booktabs}
\usepackage{relsize}
\let\raggedsection\centering

\theoremstyle{definition}
\newtheorem{defn}{Definition}[section]
\newtheorem{defn'}{Vorläufge Definition}[section]
\newtheorem{axiom}[defn]{Axiom}
\newtheorem{bsp}[defn]{Beispiel}

\theoremstyle{plain}

\newtheorem{prop}[defn]{Proposition}
\newtheorem{motto}[defn]{Motto}
\newtheorem{wunder}[defn]{Wunder}
\newtheorem{ueberlegung}[defn]{Überlegung}
\newtheorem{lemma}[defn]{Lemma}
\newtheorem{kor}[defn]{Korollar}
\newtheorem{hilfsaussage}[defn]{Hilfsaussage}
\newtheorem{satz}[defn]{Satz}
\newtheorem{thm}[defn]{Theorem}

\theoremstyle{remark}
\newtheorem{bem}[defn]{Bemerkung}
\newtheorem{warnung}[defn]{Warnung}
\newtheorem{aufg}[defn]{Aufgabe}

\definecolor{darkred}{rgb}{0.7,0,0}
\definecolor{shadecolor}{rgb}{.95,.95,.95}

\newcommand{\defeq}{\vcentcolon=}
\newcommand{\prim}[1]{\text{\textnormal{$#1$ prim}}}
\newcommand{\bigsum}{\mathop{\mathlarger{\mathlarger{\sum}}}}
\newcommand{\bigprod}{\mathop{\mathlarger{\mathlarger{\prod}}}}

\DeclareMathOperator{\ld}{ld}
\newcommand{\RR}{\mathbb{R}}

\usepackage{mathpazo}

\title{de-Rham-Kohomologie: ein erster Einblick}
\author{für den JGW-Kurs \emph{Wie ward Licht?}}

\begin{document}

\maketitle

Kohomologietheorien gehören zu den wichtigsten Werkzeugen der reinen
Mathematik. Sie kommen in verschiedenen Ausprägungen in vielen Gebieten der
Mathematik vor, unter anderem in der algebraischen Topologie, der
Differentialgeometrie, der algebraischen Geometrie, der Funktionentheorie und
der homologischen Algebra. Wenn man möchte, kommt man mit Kohomologietheorien
üblicherweise ab dem vierten oder fünften Studiensemester in Berührung; wenn
man sich aber eher für angewandte Mathematik interessiert, kann man sie auch
umgehen.

Die hauptsächlich in Differentialgeometrie verwendete Variante von Kohomologie
heißt \emph{de-Rham-Kohomologie}. Deren Grundlagen sollen hier dargelegt
werden.


\section{Erinnerung an Differentialformen}

Der "`magische $d$-Operator"' macht aus einer~$k$-Form eine~$(k+1)$-Form. An
wichtigen Rechenregeln muss man sich nur die vier folgenden merken:

\begin{enumerate}
\item Ist~$f$ eine Funktion (also eine~$0$-Form), so gilt
\[ df = d(f) = \frac{\partial f}{\partial x} dx +
  \frac{\partial f}{\partial y} dy +
  \frac{\partial f}{\partial z} dz. \]
Wenn~$f$ nicht nur von~$x$, $y$ und~$z$, sondern noch von weiteren Variablen
abhängt, so muss die Summe entsprechend ergänzt werden.
\item Für alle Differentialformen~$\omega$ gilt~$d(d(\omega)) = 0$. Kürzer,
aber etwas verwirrend, schreibt man auch: "`$d^2 = 0$"'.
\item Der $d$-Operator ist additiv, es gilt also $d(\omega + \tau) = d(\omega)
+ d(\tau)$.
\item Der $d$-Operator erfüllt die "`graduierte Produktregel": Ist~$\omega$
eine~$k$-Form und~$\tau$ eine~$\ell$-Form, so gilt
\[ d(\omega \wedge \tau) = d(\omega) \wedge \tau +
  (-1)^k \omega \wedge d(\tau). \]
\end{enumerate}

\begin{aufgabe}{Angst vor $d$?}
Vollziehe folgende Beispiele nach.
\begin{enumerate}
\item $d(x^2 - yz) = 2x\,dx - z\,dy - y\,dz$
\item $d(e^x + \sin(z^2)) = e^x\,dx + 2z\cos(z^2)\,dz$
\item $d(x \wedge dy) = dx \wedge dy$
\item $d(f(x)\,dy) = f'(x)\,dx\wedge dy$
\end{enumerate}
\end{aufgabe}

\begin{aufgabe}{Trivialität höherer Differentialformen}
Mache dir klar: Wenn als einzige Variable~$x$ vorkommen darf, so ist die
"`Nullform"' die einzige~$2$-Form.

Etwas präziser ausgedrückt: Alle~$k$-Formen auf~$\RR^1$ mit~$k \geq 2$ sind
Null.
\end{aufgabe}


\section{Geschlossene und exakte Differentialformen}

Eine Differentialform~$\omega$ heißt genau dann \emph{geschlossen},
wenn~$d\omega = 0$. Eine Differentialform~$\omega$ heißt genau dann
\emph{exakt}, wenn es eine weitere Differentialform~$\alpha$ mit~$d\alpha =
\omega$ gibt. Jede exakte Differentialform ist auch geschlossen, aber es gibt
geschlossene Differentialformen, die nicht exakt sind. Wir werden sehen, dass
de-Rham-Kohomologie gerade die Größe dieses Defekts misst.

\begin{aufgabe}{Beispiele und Nichtbeispiele für geschlossene
Differentialformen}
Zeige, dass die folgenden Differentialformen jeweils geschlossen sind:
\begin{enumerate}
\item $dx$
\item $\sin(x)\,dx$
\item $47$ (konstante Funktion, also eine $0$-Form)
\item $x^2 e^{yz}\,dx \wedge dy \wedge dz$
\end{enumerate}
Zeige, dass die folgenden Differentialformen jeweils nicht geschlossen sind:
\begin{enumerate}
\addtocounter{enumi}{4}
\item $\sin(x)$ (eine $0$-Form)
\item $-y\,dx + x\,dy$
\end{enumerate}
\end{aufgabe}

\begin{aufgabe}{Aus Exaktheit folgt Geschlossenheit}
Zeige, dass jede exakte Differentialform~$\omega$ auch geschlossen ist.
\end{aufgabe}

\begin{aufgabe}{Wegunabhängigkeit bei geschlossenen Differentialformen}
Seien~$\Gamma_1$ und~$\Gamma_2$ zwei Kurven mit gleichem Start- und Endpunkt
(zum Beispiel in der Ebene). Sei~$\omega$ eine geschlossene~$1$-Form. Zeige,
dass~$\int_{\Gamma_1} \omega = \int_{\Gamma_2} \omega$.

\emph{Tipp.} Die Differenz aus linker und rechter Seite lässt sich ebenfalls
als Integral schreiben, wobei der Integrationsweg der Rand eines
zweidimensionalen Gebiets ist. Verstehe diese kryptische Bemerkung und verwende
dann den Satz von Stokes.
\end{aufgabe}


\section{Die Kohomologie des $\RR^1$}

Die reelle Zahlengerade~$\RR^1$ ist "`kohomologisch trivial"': Jede
geschlossene~$1$-Form~$\omega$ auf~$\RR^1$ ist exakt. (Ebenfalls sind
alle geschlossenen~$2$-, $3$- und auch alle höheren Formen exakt. Das ist aber
keine interessante Aussage -- wieso?)

Das ist etwas Besonderes. Für kompliziertere geometrische Objekte als~$\RR^1$
-- insbesondere solche, die Löcher enthalten, wie etwa die zweidimensionale
Ebene mit entferntem Ursprung -- stimmt die Aussage nämlich nicht.

\begin{aufgabe}{Exaktheit aller~$1$-Formen auf~$\RR^1$}
Sei~$\omega = f(x)\,dx$ eine beliebige~$1$-Form auf~$\RR^1$. Da die
einzige~$2$-Form auf~$\RR^1$ die Nullform ist, ist~$d\omega = 0$. Wir
definieren
\[ g(x) \defeq \int_0^x f(t)\,dt. \]
Zeige: $\omega = dg$.
\end{aufgabe}


\section{Die Kohomologie des $\RR^2$}

Auch die Zahlenebene~$\RR^2$ ist kohomologisch trivial. Das ist Gegenstand der
folgenden zwei Aufgaben.

\begin{aufgabe}{Exaktheit aller geschlossenen~$2$-Formen auf~$\RR^2$}
Sei~$\omega = f(x,y)\,dx \wedge dy$ eine beliebige~$2$-Form auf~$\RR^2$.
\begin{enumerate}
\item Zeige: Die Form~$\omega$ ist automatisch geschlossen.
\item Wir definieren die~$1$-Form
\[ \alpha \defeq \left(\int_0^x f(t,y)\,dt\right) dy. \]
Zeige: $\omega = d\alpha$.
\end{enumerate}
\end{aufgabe}

\begin{aufgabe}{Exaktheit aller geschlossenen~$1$-Formen auf~$\RR^2$}
Sei~$\omega = f(x,y)\,dx + g(x,y)\,dy$ eine beliebige~$1$-Form auf~$\RR^2$.
\begin{enumerate}
\item Zeige: Genau dann ist~$\omega$ geschlossen, wenn
$\frac{\partial g}{\partial x} - \frac{\partial f}{\partial y} = 0$.
\item Wir definieren
\[ h(x,y) \defeq \int_0^x f(t,y)\,dt + \int_0^y g(x,t)\,dt. \]
Zeige: $\omega = dh$.
\end{enumerate}
\end{aufgabe}


\section{Die Kohomologie des $\RR^2 \setminus \{0\}$}

Unser erstes Beispiel für eine geometrische Figur, die nicht kohomologisch
trivial ist, ist die "`gelochte Ebene"': die Zahlenebene~$\RR^2$, aus der der
Ursprung entfernt wurde. Auf dieser gibt es Differentialformen, welche zwar
geschlossen, aber nicht exakt sind.

\begin{aufgabe}{Eine nicht-exakte Differentialform}
Wir betrachten die Differentialform
\[ \omega \defeq \frac{-y}{x^2+y^2} dx + \frac{x}{x^2+y^2} dy. \]
Beim Punkt~$(0|0)$ ist~$\omega$ nicht definiert, daher ist~$\omega$ nur eine
Differentialform auf~$\RR^2\setminus\{0\}$ und nicht eine auf ganz~$\RR^2$.

\begin{enumerate}
\item Zeige: Die Form~$\omega$ ist geschlossen.
\item Berechne das Integral von~$\omega$ über den Einheitskreis. (Zur
Kontrolle: Das Ergebnis ist~$2\pi$.)
\item Zeige: Die Form~$\omega$ ist nicht exakt. (Nutze den Satz von Stokes und
die vorherige Teilaufgabe.)
\end{enumerate}
\end{aufgabe}

Damit eine~$1$-Form auf~$\RR^2\setminus\{0\}$ exakt ist, genügt es nicht, dass
sie lediglich geschlossen ist. Stattdessen muss auch noch ihr Integral über die
Einheitskreislinie verschwinden (Null sein). Man sagt auch: \emph{Eine Basis
der ersten Homologie von~$\RR^2\setminus\{0\}$ besteht aus dem Einheitskreis.}

\begin{aufgabe}{Eine nicht-exakte Differentialform}
Sei~$\omega$ eine geschlossene~$1$-Form auf~$\RR^2\setminus\{0\}$, für die das
Integral über die Einheitskreislinie verschwindet. Wir definieren die Funktion
\[ h(x,y) \defeq \int_\Gamma \omega, \]
wobei~$\Gamma$ \emph{irgendeine} Kurve von~$(1|0)$ nach~$(x|y)$ ist, welche
nicht den Ursprung passiert.
\begin{enumerate}
\item Zeige: Die Definition von~$h(x,y)$ ist unabhängig von der Wahl der
Kurve~$\Gamma$. Andere Kurven mit selben Start- und Endpunkt führen also zum
gleichen Integralwert. (Verwende dazu die Geschlossenheit von~$\omega$ und den
Satz von Stokes.)
\item Weise die Exaktheit von~$\omega$ nach, indem du beweist,
dass~$\omega = dh$.
\end{enumerate}
\end{aufgabe}


\section{Weitere Beispiele für triviale und nichttriviale Kohomologie}

Versuche, geometrische Intuition für die folgenden Behauptungen zu gewinnen und
sie vielleicht sogar zu beweisen.

\begin{enumerate}
\item Eine~$1$-Form auf dem gelochten dreidimensionalen
Raum~$\RR^3\setminus\{0\}$ ist schon exakt, wenn sie nur geschlossen ist.
\item Eine~$2$-Form auf dem gelochten dreidimensionalen 
Raum~$\RR^3\setminus\{0\}$ ist genau dann exakt, wenn sie geschlossen ist und
wenn ihr Integral über die Einheitskugeloberfläche verschwindet.
\item Eine~$1$-Form auf der zweifach gelochten Ebene (also der Zahlenebene, aus
der zwei Punkte entfernt wurden) ist genau dann exakt, wenn sie geschlossen ist
und wenn die Integrale über kleine Kreise um die beiden Löcher verschwinden.
\item Wir entfernen aus dem dreidimensionalen Raum die~$z$-Achse sowie den in der~$x$-$y$-Ebene
liegenden Kreis mit Mittelpunkt~$(0|0|0)$ und Radius~$2$. Eine~$1$-Form auf
diesem Gebilde ist genau dann exakt, wenn sie geschlossen ist und wenn ihre
Integrale über die folgenden beiden Kurven verschwinden: der Einheitskreis in
der~$x$-$y$-Ebene und der in der~$x$-$z$-Ebene liegende Kreis mit
Mittelpunkt~$(2|0|0)$ und Radius~$1$.
\item Eine~$2$-Form auf demselben Gebilde ist genau dann exakt, wenn sie
geschlossen ist und wenn ihr Integral über die Oberfläche eines bestimmten
Torus (welchen?) verschwindet.
\end{enumerate}

\end{document}
